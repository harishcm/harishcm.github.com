\documentclass[a4paper]{article}
\usepackage[margin=2.5cm]{geometry}
\usepackage{setspace}
  \onehalfspacing

\title{Summary of ``Reliability assessment of high-strength concrete columns''
       by Sofia M. C. Diniz and Dan M. Frangopol
      }
\author{Mohanadas Harish Chandar, U067314J}

\begin{document}
\maketitle

\section*{Introduction}
High-strength concrete (HSC) is different from normal-strength concrete (NSC)
in that it has a more linear response, less microcracking, higher elastic 
modulus, higher tensile strength, less creep, and less ductility.
It is primarily used in the lower columns of high-rise buildings.
Although considerable understanding about HSC exists, there is still 
much concern about the use of HSC because of the following issues:

There is concern about HSC's strength consistency as its production
depends on low water/cement ratios, better aggregates, higher cement
contents, chemical admixtures, and strict quality controls.
The in-situ to cylinder-strength ratios used in design of HSC may need
to be lower than that used for NSC.

There is concern about HSC's brittleness as HSC is widely used in the
lower columns of high-rise buildings. These columns are subjected to loads
with small eccentricities, making brittle failures unavoidable.
Although this can be improved with adequate lateral steel detailing,
confinement is less effective for HSC than for NSC.

There is also concern when HSC is used in the construction of columns in upper
floors as only minimum steel is provided and cross-sections are
reduced. The resulting large load eccentricities may lead to instability 
problems.

As design recommendations in the American Concrete Institute (ACI) codes
were only based on concrete with strength up to 42 MPa, the authors
wanted to investigate the reliability of HSC columns in order to make
design and code recommendations for HSC columns.


\section*{Columns designed}
36 HSC columns were designed according to the ACI code. Within them they had
2 strengths---62.1 MPa and 96.5 MPa. 12 NSC columns of strength 34.5 MPa
were also designed according to the ACI code to provide benchmark reliability
values.
Dimensions of all columns were 508 mm by 508 mm with a cover of 38.1 mm.
The columns were bent in a single curvature with equal moments at both ends, and
design loads were set to match the design strength of the columns exactly.

Among the columns, there were 2 longitudinal steel reinforcement ratios---0.013
and 0.038, three slenderness ratios---0, 22 \& 50, and three degrees of 
confinement---unconfined, minimum confinement according to the ACI code and more
than the minimum confinement. For all columns, the mean live load to mean dead
load ratios were set to 1, except when the effect of mean load ratio 
itself was investigated.


\section*{Reliability Analysis}
RC columns do not have any closed-form expressions to describe column strength.
Neither is there an unique formulation to define the failure criteria. 
In their reliability analysis, the authors chose a performance function 
postulated by Israel et al., 1987:

\begin{equation}
g(\mathbf{X}) = \left[ P^2 + \left( \frac{P \cdot e}{h} \right) ^2 \right] ^{1/2} - \left\{\left(D + L\right)^2 + \left[\frac{\left(D + L\right)e}{h}\right]^2\right\}^{1/2}
\end{equation}
where $P$ = axial resistance of the column at a given eccentricity $e$; 
$h$ = cross section height; and $D$ and $L$ =, respectively, dead and live
load acting on the column.

The strength statistics at different $e/h$ ratios were generated using Monte Carlo
simulations in previous studies by the authors (Diniz, 1994; Diniz and Frangopol, 1997b).
The magnified moments at higher eccentricities were taken into account
by reducing the strength of column, instead of increasing the loading.
The load statistics used were also computed in previous studies by the authors
(Diniz, 1994; Diniz and Frangopol, 1997a).

Both FORM and SORM were used in the actual determination of reliability indices.
It was found that there was negligible difference between the two methods, and
only the FORM values were reported.

\section*{Findings}
The reliability analysis found that only small or negligible effects were caused 
when varying the amount of confining steel. However, significant effects were caused 
when varying concrete compressive strength, column slenderness ratio and 
amount of longitudinal steel.

It was found that HSC columns were generally less reliable than NSC columns.
The lowest reliability levels ($\beta \leq$ 2.5) were produced when columns 
had a combination of high concrete compressive strength, high slenderness ratio
and minimum amount of longitudinal steel. 
The worst case in the investigation occurred when the 96.5 MPa column with a 
slenderness ratio of 50 and a minimum amount of longitudinal steel was loaded
with an $e/h$ ratio of 0.7.
At small eccentricities however, the reliability of HSC slender columns was
higher than that of the corresponding HSC short columns.

In HSC short columns, it was found that adequate reliability levels may 
be achieved by lowering the in-situ to cylinder-strength ratio, and by keeping
the ratio of load carried by longitudinal steel to load carried by
concrete section constant, even as concrete strength was increased.

In HSC slender columns, it was found that adequate reliability levels may
be achieved by increasing the minimum amount of longitudinal steel in proportion
to the increase in concrete compressive strength.  


\section*{Conclusion}
For HSC short columns, it was concluded that their lower reliabilities as
compared to that of NSC short columns was primarily caused by the lower 
in-situ to cylinder-strength ratio associated with HSC.

For HSC slender columns, it was concluded that low reliabilities 
(say $\beta \leq$ 3.3) were primarily due to the relatively small amounts of 
longitudinal steel in these columns.

The ACI code was concluded as being adequate for designing HSC columns, provided
that: a lower in-situ to cylinder-strength ratio was assumed during design, and
in the case of slender columns, the minimum allowable amount of longitudinal
steel was increased in proportion to the increase in concrete compressive strength.

\section*{References}
\begin{list}{}{\setlength{\leftmargin}{0pt}}

\item Israel, M., Ellingwood, B., and Corotis, R. (1987). ``Reliability-based
code formulation for reinforced concrete buildings'', {\it J. Struct. Div.},
ASCE, 113(10), 2235--2252

\item Diniz, S. M. C. (1994). ``Reliability evaluation of high strength concrete columns'',
PhD dissertation, Dept. of Civ., Envir., and Arch. Engrg., Univ. of Colorado, Boulder, Colo.

\item Diniz, S. M. C., and Frangopol, D. M. (1997a). ``Reliability bases for high-strength
concrete columns'', {\it J. Struct. Engrg.}, ASCE, 123(10), 1375--1381

\item Diniz, S. M. C., and Frangopol, D. M. (1997b). ``Strength and ductility simulation of
high-strength concrete columns'', {\it J. Struct. Engrg.}, ASCE, 123(10) 1365--1374

\item Diniz, S. M. C., and Frangopol, D. M. (1998). ``Reliability assessment 
of high-strength concrete columns'', {\it J. Engrg. Mech.}, ASCE, 124(5), 529--536

\end{list}

\end{document}
